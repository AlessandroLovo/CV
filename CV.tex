\documentclass[11pt, a4 paper]{article}

\usepackage[top=1cm,bottom=2cm,left=2cm,right=2cm]{geometry}
\usepackage[english]{babel}
\usepackage[T1]{fontenc}
\usepackage[utf8]{inputenc}
\usepackage{url}
\usepackage{natbib}
% \usepackage{apacite}
\usepackage{multicol}
\usepackage{enumitem}

\usepackage{hyperref}


\renewcommand{\bibsection}{\section*{Full references to publications}}
\newcommand{\bu}{$\bullet$ }

\begin{document}

\title{Alessandro Lovo}
% \date{\emph{phone:} 0039 347 2462687, \emph{e-mail:} ale571.lovo@gmail.com}
% \author{Born January 7 1998 in Vicenza, Italy}
\date{}
\author{}
\maketitle


\section*{Personal details}
  \begin{multicols}{2}
    \begin{description}[style=multiline,leftmargin=2cm,align=right]
      \item[birth] January 7, 1998
      \item[address] Via Arzignano, 19, Vicenza, Italy
      % \item[address] 3 Rue Ornano, 69001, Lyon, France
      % \item[phone] 0039 347 2462687
      \item[phone] 0033 (0)6 08 06 39 17
      \item[e-mail]
        % ale571.lovo@gmail.com \\
        alessandro.lovo.@ens-lyon.fr

      \item[GitHub] {\small \url{https://github.com/AlessandroLovo}}
      \item[Website] {\small \url{https://alessandrolovo.github.io/}}
    \end{description}
  \end{multicols}



\section*{Education}
  \begin{description}[style=multiline,leftmargin=3cm,align=right]
    \item[2021-2024]
      PhD @ \emph{\'Ecole Normale Supérieure de Lyon} as H2020 MSCA Fellow within the \emph{CriticalEarth} ITN \\
      Thesis title: Studying extreme climate events using machine learning and rare event algorithms. \\
      Defense date: October 15, 2024 \\
      During my PhD I did two 3-month Secondments: one at the University of Copenhagen and the other at the University of Utrecht.
    \item[2016-2022]
      \emph{Scuola Galileiana di Studi Superiori, Padova}. Honors college with scholarship: over the five years of Bsc and Msc, students attend 12 courses and several seminars on various topics such as \emph{3D printing} or \emph{Quantum Optics}. \\ For more info see \url{https://scuolagalileiana.unipd.it/} \\
      Thesis title: Interpretability of deep learning probabilistic prediction of extreme heatwaves over France \\
      Final grade: 110/110 with honors \\
      \emph{Note: the thesis was based on the work of the first year of PhD.}
    \item[2019-2021]
      Master Degree in Physics (of Matter) @ \emph{Università degli Studi di Padova} \\
      Thesis title: Control of the emission properties of quantum emitters by coupling with phase-change nanomaterials \\
      Final grade 110/110 with honors
    \item[2016-2019]
      Bachelor Degree in Physics @ \emph{Università degli Studi di Padova} \\
      Thesis title: Automatic classification of particles in a Cloud Chamber \\
      Final grade: 110/110 with honors
    % \item[2011-2016]
    %   \emph{Liceo Scientifico Statale P. Lioy, Vicenza}. Final grade: 100/100 with honors
    % \item[2008-2011] \emph{Scuola Secondaria di primo Grado F. Maffei, Vicenza}. Final grade 10L/10
  \end{description}


% \section*{Work experience}
%   none


\section*{Skills}
  \begin{description}[style=multiline,leftmargin=3cm,align=right]
    \item[Languages]
      Italian (Mother tongue) \\
      English (C1) \\
      French (B2) \\
      German (A2) \\
      Danish (A1)
      % Dutch (A1)
    \item[Software]
      python (including PyTorch and Tensorflow), C++, LaTeX, Excel
    \item[Transversal]
      Teamwork, creative thinking, problem-solving, critical thinking, scheduling, determination, curiosity, efficiency
  \end{description}


\section*{Awards}
  \begin{description}[style=multiline,leftmargin=3cm,align=right]
    \item[2016]
      % First place (over 336 applicants) in the rankings for the admission to the \emph{Scuola Galileiana di Studi Superiori} \\
      Bronze Medal @ International Physics Olympiad \\
      Gold Medal @ Italian Physics Olympiad
    \item[2015] Bronze Medal @ Italian Physics Olympiad
  \end{description}


\section*{Research interests}
  \begin{description}[style=multiline,leftmargin=3cm,align=right]
    \item[] Climate dynamics, climate change, machine learning, neural networks, complex systems, renewable energies, green transition, nanophysics, space exploration, nuclear fusion
  \end{description}


\section*{Other interests}
  \begin{description}[style=multiline,leftmargin=3cm,align=right]
    \item[Sports] Sailing, bouldering, skiing, competitive chess
    \item[Music] Piano, guitar, composing/arranging
    \item[Other] Painting, creative writing, cooking, woodworking
  \end{description}

\section*{Publications}
  \begin{description}[style=multiline,leftmargin=3cm,align=right]
    \item[2024]
    \bu IN PRESS to Artificial Intelligence for the Earth Systems: \emph{Tackling the Accuracy-Interpretability Trade-off in a Hierarchy of Machine Learning Models for the Prediction of Extreme Heatwaves} \cite{lovoTacklingAccuracyInterpretabilityTradeHierarchy2024} \\
    \bu SUBMITTED to Proceedings of the Royal Society A: \emph{The role of edge states for early-warning of tipping points} \cite{lohmannRoleEdgeStates2024} \\
    \bu SUBMITTED to Journal of Advances in Modeling Earth Systems: \emph{Gaussian Framework and Optimal Projection of Weather Fields for Prediction of Extreme Events} \cite{mascoloGaussianFrameworkOptimal2024} (I am joint first author)
    \item[2023]
    \bu \emph{Active Modulation of Er3+ Emission Lifetime by VO2 Phase-Change Thin Films} \cite{kalinicActiveModulationEr32024} (publication resulting from my Master Thesis)
  \end{description}

\section*{Conferences}
  \begin{description}[style=multiline,leftmargin=3cm,align=right]
    \item[2024]
      \bu April 14-19, Wien, Austria: \textbf{EGU}, talk: \emph{A Gaussian framework for optimal prediction of extreme heat waves} (presented by F. Bouchet) \cite{mascoloGaussianFrameworkOptimal2024a}
    \item[2023]
      \bu October 2-6, Paris, France: \textbf{Group de recherche "Défis Théoriques pour les sciences du climat: Workshop Previsibilité et points de bascule}, poster: \emph{Heat waves prediction using Gaussian assumption} \\
      \bu June 5-17, Saint Pierre Quiberon, France: \textbf{Beg Rohu Summer School 'Wind and Physics'}, poster: \emph{Heat waves prediction using Gaussian assumption} \\
      \bu April 24-28, Wien, Austria: \textbf{EGU}, talk: \emph{Interpretable probabilistic forecast of extreme heat waves} \cite{lovoInterpretableProbabilisticForecast2023} \\
      \bu February 22-24, Muenchen, Germany: \textbf{Joint CriticalEarth-TiPES conference}, PICO talk: \emph{Trying to tip the AMOC with Rare Event Algorithms}
    \item[2022]
      \bu December 14, Paris, France: \textbf{ANR SAMPRACE meeting}, talk: \emph{Interpretability of deep learning probabilistic prediction of extreme heatwaves over France}
  \end{description}


\section*{References}
\begin{description}[style=multiline,leftmargin=4cm,align=right]
  \item[Freddy Bouchet]
    % Professor @
    \emph{CNRS, \'Ecole Normale Sup\'erieure, IPSL, Laboratoire de M\'et\'eorologie Dynamique} \\
    freddy.bouchet@lmd.ipsl.fr
  \item[Corentin Herbert]
    % Chargé de Recherche - HDR @
    \emph{\'Ecole Normale Sup\'erieure de Lyon, Laboratoire de Physique} \\
    corentin.herbert@ens-lyon.fr
  \item[Johannes Lohmann]
    % Assistant Professor @
    \emph{Physics of Ice, Climate and Earth, Niels Bohr Institute,
    University of Copenhagen, Denmark} \\
    johannes.lohmann@nbi.ku.dk
  % \item[Marco Zanetti]
    % Associate professor @
  %   \emph{Università degli Studi di Padova} (BsC thesis supervisor) \\
  %   marco.zanetti@cern.ch
  % \item[Alberto Testolin]
  %   Researcher @
  %   \emph{Università degli Studi di Padova}
  %   (Professor of the \emph{Neural Networks and Deep Learning} course) \\
  %   alberto.testolin@unipd.it
  % \item[Tiziana Cesca]
  %   Associate professor @
  %   \emph{Università degli Studi di Padova} (MsC thesis supervisor) \\
  %   tiziana.cesca@unipd.it

\end{description}

% \bibliographystyle{apalike}
\bibliographystyle{plainnat}
\bibliography{papers}


\end{document}
